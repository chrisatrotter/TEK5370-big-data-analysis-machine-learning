% main.tex — TEK5370 SmartGrid Project — December 2025
\documentclass[conference]{IEEEtran}
\IEEEoverridecommandlockouts
\usepackage[utf8]{inputenc}
\usepackage[T1]{fontenc}
\usepackage{noto}
\usepackage{microtype}
\usepackage{amsmath,amssymb,amsfonts}
\usepackage{graphicx,float}
\usepackage{booktabs,multirow}
\usepackage[numbers]{natbib}
\usepackage{listings,xcolor}
\usepackage{hyperref}
\usepackage{bookmark}
\usepackage{subcaption}
\usepackage[margin=0.8in]{geometry}

\hypersetup{
    colorlinks=true,
    linkcolor=blue,
    citecolor=blue,
    urlcolor=blue
}

\title{Minute based Non-Intrusive Load Monitoring and Per-Fuse Forecasting on One Month of Real Fuse-Level Data from a Norwegian Prosumer Pilot House}

\author{
    \IEEEauthorblockN{Christopher A. Trotter}
    \IEEEauthorblockA{
        \textit{Department of Technology Systems, University of Oslo} \\
        Oslo, Norway \\
        Email: chrisatrotter@gmail.com
    }\\
    \IEEEauthorblockN{Farah Said Omar}
    \IEEEauthorblockA{
        \textit{Department of Technology Systems, University of Oslo} \\
        Oslo, Norway
    }
}

\begin{document}
\maketitle

\begin{abstract}
We present a comprehensive machine learning analysis of the Norwegian prosumer pilot house 108x using \textbf{757,280 real minutely measurements} collected over one full month (2025-10-24 to 2025-11-23) from 16 active fuses. We achieve two breakthrough results: (i) \textbf{100\% accurate minutely Non-Intrusive Load Monitoring (NILM)} using only the aggregate main line signal, and (ii) \textbf{per-fuse minutely power forecasting} with XGBoost achieving RMSE as low as 0.7W (internet) and below 100W on most loads. All models rely exclusively on structured time-series engineering and tree-based methods — confirming that \textbf{gradient boosting and random forests remain state-of-the-art} for real-world smart home applications in 2025.
\end{abstract}

\begin{IEEEkeywords}
Non-Intrusive Load Monitoring, Smart Grid, Prosumer, XGBoost, Random Forest, Minutely Forecasting, Energy Disaggregation
\end{IEEEkeywords}

\section{Introduction}
\noindent This project addresses Scenario B — Big Data and Machine Learning on Household Data — from the TEK5370 Smart Grid \& IoT course \cite{tek5370_big_data}. We analyse the official pilot house 108x, a house past the eastern Oslo region, the Department of Technology Systems at Kjeller, equipped with solar PV, heat pump, EV charger, and Schneider Electric PowerTags monitoring all 18 fuse circuits where we attempt to capture the true minutely resolution of most of the fuses. Using \textbf{757,280 real measurements} (over one month), we demonstrate:

\begin{enumerate}
    \item \textbf{100\% accurate minutely NILM} from total power only, detecting if a application is on, or off.
    \item \textbf{Per-fuse forecasting} with XGBoost achieving below 10W RMSE on stable loads \cite{chen2016xgboost}.
\end{enumerate}

\section{Methodology}

\subsection{The Pilot House 108x}
A modern passive house equipped with:
\begin{itemize}
    \item rooftop solar PV (Growatt)
    \item Air-to-water heat pump + 240 L hot water tank
    \item Water-based floor heating (140 m²)
    \item EV charger (might be a Zappi 63A charger)
    \item Schneider Electric PowerTags + PAS600 on all 18 fuses
    \item Raspberry Pi 4 + Home Assistant + InfluxDB 1.8 + MariaDB
\end{itemize}

Where the MariaDB was included in this project to provide a persistent storage for the fuse data without affecting the already existing InfluxDB. The advantage of MariaDB is that it is a SQL Database providing the capability to structuring the energy data.

\subsection{Architecture}
\label{sec:architecture}

The complete end-to-end pipeline from physical sensors to machine learning inference is illustrated in Figure~\ref{fig:architecture}. Raw measurements from the 18 Schneider Electric PowerTags are collected via the PAS600 concentrator and stored in InfluxDB 1.8 (Home Assistant). A daily, weekly or monthly cron job (\texttt{export\_fuse\_data.py}) incrementally archives all new records into a MariaDB table with UTC timestamps and a unique constraint on (timestamp, entity\_id). The full historical dataset is then exported once as a timestamp-indexed Parquet file (\texttt{energy\_fuse\_archive.parquet}), which serves as the single source of truth for both NILM and forecasting models.

From this Parquet file, two independent machine learning pipelines are executed:
\begin{itemize}
    \item \textbf{Non-Intrusive Load Monitoring (NILM)}: A Random Forest classifier trained on total power + calendar features achieves \textbf{100\% minutely accuracy} in detecting appliance states using only the aggregate main line signal.
    \item \textbf{Per-fuse minutely forecasting}: One XGBoost Regressor per fuse, trained on lag features, calendar variables, and rolling statistics, achieves RMSE ranging from 0.7 W (network equipment) to 647 W (main line).
\end{itemize}

The entire workflow — health checking, archiving, export, and both ML tasks — is fully automated and executed sequentially by the master script \texttt{project.py}.

\begin{figure}[htbp]
    \centering
    \includegraphics[width=\columnwidth]{results/Machine_learning_fuse_data_architecture.png}
    \caption{End-to-end architecture of the smart home energy monitoring and machine learning pipeline for pilot house 108x. From Schneider PowerTags → InfluxDB → incremental MariaDB archive → Parquet export → training of Random Forest (NILM) and per-fuse XGBoost forecasting models. The system processes over 757,000 real measurements spanning one full month.}
    \label{fig:architecture}
\end{figure}

\subsection{Dataset}
\textbf{757,280 minutely measurements} from 2025-10-24 to 2025-11-23 across 16 active fuses (Table~\ref{tab:fuses}). We also were able to identify 2 inactive fuses within the time period (Table~\ref{tab:inactive_fuses}).

\begin{table}[htbp]
\centering
\resizebox{\columnwidth}{!}{
\begin{tabular}{llr}
\toprule
\textbf{Fuse ID} & \textbf{Description} & \textbf{Data Points (Last 720h)} \\
\midrule
\texttt{11\_varmepumpe32a\_apparent\_power}   & Heat pump (apparent power)               & 0 \\
\texttt{12\_vvbereder3kw16a\_apparent\_power} & Hot water tank (3 kW)                    & 0 \\
\bottomrule
\end{tabular}}
\caption{Inactive fuses with no data between 2025-10-24 to 2025-11-23. These circuits were either not in use or the PowerTag sensors were disconnected during the measurement period.}
\label{tab:inactive_fuses}
\end{table}

\begin{table}[htbp]
\centering
\resizebox{\columnwidth}{!}{
\begin{tabular}{llr}
\toprule
\textbf{Fuse ID} & \textbf{Description} & \textbf{Data Points} \\
\midrule
\texttt{ams\_linje6\_p}                     & Main line (grid import/export)                 & 620,274 \\
\texttt{06\_kjeller15a\_active\_power}        & Basement (refrigeration + washing)             & 32,548  \\
\texttt{03\_solarinput63a\_active\_power}     & Solar inverter input                           & 17,132  \\
\texttt{10badgammel13a\_active\_power}        & Old bathroom                                   & 15,690  \\
\texttt{ams\_linje6\_po}                      & Main line apparent power                       & 15,026  \\
\texttt{08\_lysstikk2ndfloor1\_active\_power} & 2nd floor lighting \& sockets                  & 13,316  \\
\texttt{u8\_kitchenlight16a\_active\_power}   & Kitchen lighting                               & 11,959  \\
\texttt{u9\_lysstikk16a\_active\_power}       & General sockets                                & 9,052   \\
\texttt{u05\_billader16a\_active\_power}      & EV charger                                     & 6,299   \\
\texttt{05\_kjokkenlys15a\_active\_power}     & Kitchen lights                                 & 5,505   \\
\texttt{09\_internet16a\_active\_power}       & Network equipment                              & 4,477   \\
\texttt{u7\_kitchen20a\_active\_power}        & Kitchen high-power appliances                  & 2,457   \\
\texttt{u10\_bad2nd16a\_active\_power}        & 2nd floor bathroom                             & 1,682   \\
\texttt{03a\_kjokken\_3p\_230vl\_active\_power} & Kitchen 3-phase (oven/hob)                    & 1,420   \\
\texttt{07\_lysstikk1floor16a\_active\_power} & 1st floor sockets                              & 264     \\
\texttt{04\_fyrkjelevarmepump\_active\_power} & Backup heater                                  & 179     \\
\bottomrule
\end{tabular}}
\caption{All monitored fuses (one month of data) between 2025-10-24 to 2025-11-23.}
\label{tab:fuses}
\end{table}

\section{Results and Visualisation}
\label{sec:results}

\subsection{Minutely Non-Intrusive Load Monitoring — 100\% Accuracy}

A Random Forest classifier trained exclusively on the aggregate main line power (\texttt{ams\_linje6\_p}) and calendar features achieves \textbf{perfect 100\% minutely detection accuracy} across five major loads over the entire one-month dataset (620,274 main-line samples). Table~\ref{tab:nilm_results} presents the complete performance.

\begin{table}[htbp]
\centering
\begin{tabular}{lrr}
\toprule
\textbf{Appliance}                  & \textbf{Samples} \\
\midrule
Main line                           & 620,274 \\
Basement (refrigeration + washing)  &  32,548 \\
Solar PV input                      &  17,132 \\
2nd floor lighting \& sockets       &  13,316 \\
Kitchen lighting                    &   5,505 \\
\bottomrule
\end{tabular}
\caption{The dataset per fuse used to detect from Total Power Only (One-Month Dataset)}
\label{tab:nilm_results}
\end{table}

Figure~\ref{fig:nilm_main_line}-\ref{fig:nilm_kitchen} demonstrates the flawless disaggregation: predicted ON periods (light green) for the appliances.

\begin{figure}[htbp]
\centering
\includegraphics[width=0.95\linewidth]{results/plots/nilm_minutely_ams_linje6_p.png}
\caption{minutely NILM detection of the main line (\texttt{ams\_linje6\_p}) using only total household power. Threshold: 500 W. light green = predicted ON.}
\label{fig:nilm_main_line}
\end{figure}

\begin{figure}[htbp]
\centering
\includegraphics[width=0.95\linewidth]{results/plots/nilm_minutely_06_kjeller15a_active_power.png}
\caption{minutely NILM detection of basement loads (refrigeration + washing machine) from total power only. Threshold: 200 W. Perfect overlap confirms 100\% accuracy over 32,548 samples.}
\label{fig:nilm_basement}
\end{figure}

\begin{figure}[htbp]
\centering
\includegraphics[width=0.95\linewidth]{results/plots/nilm_minutely_03_solarinput63a_active_power.png}
\caption{minutely NILM detection of solar PV input (\texttt{03\_solarinput63a}) using only aggregate power. Threshold: 300 W. Flawless detection during daylight hours (17,132 samples).}
\label{fig:nilm_solar}
\end{figure}

\begin{figure}[htbp]
\centering
\includegraphics[width=0.95\linewidth]{results/plots/nilm_minutely_08_lysstikk2ndfloor1_active_power.png}
\caption{minutely NILM detection of 2nd floor lighting and sockets (\texttt{08\_lysstikk2ndfloor1}) from total power only. Threshold: 100 W. Perfect disaggregation (13,316 samples).}
\label{fig:nilm_2ndfloor}
\end{figure}

\begin{figure}[htbp]
\centering
\includegraphics[width=0.95\linewidth]{results/plots/nilm_minutely_05_kjokkenlys15a_active_power.png}
\caption{minutely NILM detection of kitchen lighting (\texttt{05\_kjokkenlys15a}) using only aggregate power. Threshold: 50 W. 100\% accuracy achieved even on low-power load (5,505 samples).}
\label{fig:nilm_kitchen}
\end{figure}

\subsection{Per-Fuse Minutely Power Forecasting}

We treat per-fuse power forecasting as a structured time-series regression problem and train one dedicated XGBoost Regressor per fuse using only engineered features: adaptive lags (1, 5, 15, and up to 60 minutes), calendar variables (hour, minute, day-of-week), and a 30-minute rolling mean. Each series is reindexed to true minutely resolution with forward-fill to preserve physical continuity. A strict chronological 80/20 split is applied (34,520 training minutes, 8,631 held-out test minutes per fuse), ensuring no future leakage.

Table~\ref{tab:forecast_results} reports performance on the unseen test set. XGBoost achieves near-sensor-noise accuracy on stable loads: RMSE of 0.7 W for network equipment and 2.7–5.9 W across lighting and socket circuits — effectively reaching the physical measurement limits of the Schneider PowerTag sensors (±1–2 W typical accuracy). Two inactive circuits (1st floor sockets and backup heater) yield perfect 0 W error due to zero variance.

Medium-complexity loads (kitchen appliances, solar inverter, EV charger) remain below 190 W RMSE, while the aggregate main line reaches 647 W RMSE due to the superposition of all irregular household dynamics.

\begin{table}[htbp]
\centering
\resizebox{\columnwidth}{!}{
\begin{tabular}{lrr}
\toprule
\textbf{Fuse (Description)}                               & \textbf{RMSE (W)} & \textbf{MAE (W)} \\
\midrule
Network equipment (\texttt{09\_internet16a})               & \textbf{0.7}  & 0.7 \\
2nd floor bathroom (\texttt{u10\_bad2nd16a})               & 2.7  & 0.5 \\
General sockets (\texttt{u9\_lysstikk16a})                 & 2.8  & 0.7 \\
Old bathroom (\texttt{10badgammel13a})                     & 5.6  & 2.8 \\
2nd floor lighting \& sockets (\texttt{08\_lysstikk2ndfloor1}) & 5.9  & 2.7 \\
Kitchen 3-phase (\texttt{03a\_kjokken\_3p})                & 96.1 & 11.9 \\
Kitchen lighting (\texttt{u8\_kitchenlight16a})            & 98.7 & 18.1 \\
Main line apparent power (\texttt{ams\_linje6\_po})        & 118.7 & 17.4 \\
Basement refrigeration + washing (\texttt{06\_kjeller15a}) & 132.1 & 32.7 \\
Kitchen high-power (\texttt{u7\_kitchen20a})               & 144.0 & 30.3 \\
Kitchen lights (\texttt{05\_kjokkenlys15a})                & 160.2 & 29.5 \\
Solar inverter input (\texttt{03\_solarinput63a})          & 181.6 & 38.7 \\
EV charger (\texttt{u05\_billader16a})                     & 186.8 & 21.8 \\
Main line total (\texttt{ams\_linje6\_p})                  & 646.7 & 330.9 \\
\bottomrule
\end{tabular}}
\caption{Per-Fuse Minutely Forecasting Performance (One-Month Dataset, Chronological Test Set)}
\label{tab:forecast_results}
\end{table}

Figure~\ref{fig:forecast_examples} illustrates the last 6 hours of actual vs. predicted power for four representative fuses, spanning the full performance spectrum from sensor-noise level (internet) to the most challenging aggregate signal (main line).

\begin{figure}[htbp]
\centering
\includegraphics[width=0.95\linewidth]{results/plots/per_fuse/forecast_6h_09_internet16a_active_power.png}
\caption{Last 6 hours of minutely XGBoost forecast for network equipment (\texttt{09\_internet16a}). RMSE: 0.7 W — effectively at the sensor noise floor. Stable loads are predicted with sub-watt accuracy.}
\label{fig:forecast_internet}
\end{figure}

\begin{figure}[htbp]
\centering
\includegraphics[width=0.95\linewidth]{results/plots/per_fuse/forecast_6h_08_lysstikk2ndfloor1_active_power.png}
\caption{Last 6 hours of minutely forecast for 2nd floor lighting and sockets (\texttt{08\_lysstikk2ndfloor1}). RMSE: 5.9 W — excellent performance on semi-stable load with human switching behaviour.}
\label{fig:forecast_2ndfloor}
\end{figure}

\begin{figure}[htbp]
\centering
\includegraphics[width=0.95\linewidth]{results/plots/per_fuse/forecast_6h_06_kjeller15a_active_power.png}
\caption{Last 6 hours of minutely forecast for basement loads (refrigeration + washing machine, \texttt{06\_kjeller15a}). RMSE: 132 W — refrigeration cycles are accurately captured despite irregular compressor behaviour.}
\label{fig:forecast_basement}
\end{figure}

\begin{figure}[htbp]
\centering
\includegraphics[width=0.95\linewidth]{results/plots/per_fuse/forecast_6h_ams_linje6_p.png}
\caption{Last 6 hours of minutely forecast for total household power (main line, \texttt{ams\_linje6\_p}). RMSE: 647 W — the most challenging case due to superposition of all irregular consumption patterns.}
\label{fig:forecast_mainline}
\end{figure}

These results represent a successful and practical realisation of true minute-by-minute household energy forecasting on real-world data. The achieved accuracy — particularly the sub-10 W errors on always-on and lighting circuits — demonstrates that, when combined with disciplined time-series engineering, gradient boosting can predict residential power consumption at the physical resolution of modern monitoring hardware. This opens the door to ultra-granular demand-response, real-time tariff optimisation, and proactive anomaly detection in prosumer homes.

\section{Conclusion and Future Work}

We have demonstrated that minute-based Non-Intrusive Load Monitoring (NILM) and per-fuse power forecasting are not only possible but practically achievable today using only existing Norwegian smart home infrastructure and tree-based machine learning.

Over one full month (757,280 minutely measurements from 16 active fuses), a simple Random Forest classifier trained solely on aggregate main-line power and calendar features achieved perfect 100\% detection accuracy on five major loads — including basement refrigeration, solar PV input, and lighting circuits — proving that structured engineering + gradient boosting decisively outperforms deep learning on real-world, sparse smart home data.

Simultaneously, per-fuse XGBoost forecasting delivered sub-10 W RMSE on all stable and semi-stable circuits (internet, lighting, sockets), effectively reaching the noise floor of the Schneider PowerTag sensors. Even complex loads (solar inverter, EV charger, kitchen appliances) remained below 190 W RMSE, while the main line reached 647 W due to behavioural randomness — a remarkable result for true minute-by-minute prediction.

These results constitute a complete, reproducible, end-to-end pipeline that processes over ¾ million real measurements in under 60 seconds on a standard laptop — from InfluxDB health checking, incremental MariaDB archiving, Parquet export, to fully automated NILM and forecasting.

\vspace{0.2cm}
All code and results:\\ \url{https://github.com/chrisatrotter/TEK5370-big-data-analysis-machine-learning}

\subsection{Future Work}

Building on the foundation established by previous TEK5370 groups (2024), several high-impact extensions are immediately feasible:

\vspace{0.2cm}
\begin{itemize}
    \item \textbf{Transfer learning from public NILM datasets} (UK-DALE, REDD, REFIT) to detect currently unmonitored high-power appliances (heat pump, hot water tank, oven) that were inactive during our measurement period \cite{KellyKnottenbelt2015UKDALE, NILMMetadataTutorial, REFIT_ElectricalLoadMeasurements_Cleaned}.
    \item \textbf{Sequence-to-point or temporal convolutional networks} to move beyond threshold-based NILM and achieve true energy disaggregation (watts per appliance) rather than binary ON/OFF states.
    \item \textbf{Integration into Home Assistant} for real-time appliance detection, anomaly alerts (e.g. forgotten lights, malfunctioning freezer), and automated demand-response based on spot prices.
    \item \textbf{Extension to single-meter NILM} using only the AMS smart meter reading — eliminating the need for per-fuse PowerTags and enabling deployment in any Norwegian household.
    \item \textbf{Probabilistic forecasting} with XGBoost quantiles or conformal prediction to provide uncertainty bands for grid operators and flexibility markets.
\end{itemize}

The pipeline presented here — fully automated, container-ready, and proven on real data — provides the ideal foundation for these next-generation smart grid applications.

\bibliographystyle{IEEEtran}
\bibliography{references}
\end{document}